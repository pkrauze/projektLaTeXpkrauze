\documentclass[a4paper]{article}
\usepackage[T1]{fontenc}
\usepackage[polish]{babel}
\usepackage[utf8]{inputenc}
\usepackage{lettrine}
\usepackage[left=35mm,right=35mm,top=0.8in,bottom=0.7in]{geometry}
\usepackage{graphicx} 
\usepackage{indentfirst}
\usepackage{amsmath}

\begin{document}
\title{Twierdzenie Pitagorasa}

\begin{Huge}\textbf{Twierdzenie Pitagorasa}\end{Huge}
\\\\
\lettrine[lines=1]{T}wierdzenie Pitagorasa – jest twierdzeniem geometrii euklidesowej, które w naszym
(zachodnio-europejskim) kręgu kulturowym przypisywane jest żyjącemu w VI wieku p.n.e.greckiemu matematykowi i filozofowi Pitagorasowi, chociaż niemal pewne jest, że znali je przed nim starożytni Egipcjanie. Wiadomo też, że jeszcze przed Pitagorasem znano je w starożytnych Chinach, Indiach i Babilonii.


\section{Teza}
\begin{itemize}
\item \emph{W dowolnym trójkącie prostokątnym, suma pól kwadratów zbudowanych na
przyprostokątnych trójkąta prostokątnego równa jest polu kwadratu zbudowanego na
przeciwprostokątnej tego trójkąta.}
\begin{center}
lub
\end{center}
\item \emph{W trójkącie prostokątnym suma kwadratów długości przyprostokątnych jest równa
kwadratowi długości przeciwprostokątnej tego trójkąta. }
\end{itemize}


\section{Interpretacja}

\begin{figure}[h]
\centering
\includegraphics[width=7cm]k
\caption{Interpretacja twierdzenia Pitagorasa}
\label{fig:obrazek k}
\end{figure}

\indent Oto interpretacja geometryczna: jeżeli na bokach trójkąta prostokątnego zbudujemy
kwadraty, to suma pól kwadratów zbudowanych na przyprostokątnych tego trójkąta jest
równa polu kwadratu zbudowanego na przeciwprostokątnej. 

\section{Dowody}
Liczba różnych dowodów twierdzenia Pitagorasa jest przytłaczająca, według
niektórych źródeł przekracza 350. Euklides w Elementach podaje ich osiem, kolejne
pojawiały się na przestrzeni wieków i pojawiają aż po dni dzisiejsze.
Niektóre z dowodów są czysto algebraiczne (jak dowód z podobieństwa trójkątów),
inne mają formę układanek geometrycznych (prawdopodobny dowód Pitagorasa), jeszcze
inne oparte są o równości pól pewnych figur. Zaprezentuje tu jedynie dwa wybrane, najbardziej popularne dowody:

\begin{enumerate}
  \item Dowód układanka
  \item Dowód przez podobieństwo
\end{enumerate}

\newpage

\subsection{Dowód układanka}
Dany jest trójkąt prostokątny o bokach długości \textit{a,b i c} jak rysunku z lewej.
Konstruujemy kwadrat o boku długości \textit{a + b} w sposób ukazany na rysunku z lewej, a
następnie z prawej. Z jednej strony pole kwadratu równe jest sumie pól czterech trójkątów
prostokątnych i kwadratu zbudowanego na ich przeciwprostokątnych, z drugiej zaś równe jest
ono sumie pól tych samych czterech trójkątów i dwóch mniejszych kwadratów zbudowanych
na ich przyprostokątnych. Stąd wniosek, że pole kwadratu zbudowanego na
przeciwprostokątnej jest równe sumie pól kwadratów zbudowanych na przyprostokątnych. 

\begin{figure}[h]
\centering
\includegraphics[width=7cm]l
\caption{Dowód twierdzenia Pitagorasa}
\label{fig:obrazek l}
\end{figure}

\indent Szczepan Jeleński w książce \textit{Śladami Pitagorasa} przypuszcza, że w ten sposób mógł
udowodnić swoje twierdzenie sam Pitagoras. \\
\indent Powyższy dowód (Rys.\ref{fig:obrazek l}), choć prosty, nie jest elementarny w tym sensie, że jego poprawność
wymaga uprzedniego uzasadnienia, że pole kwadratu złożonego z trójkątów i mniejszych
kwadratów jest równe sumie pól tych figur. Może się to wydawać oczywiste, jednak dowód
tego faktu wymaga uprzedniego zdefiniowania pola, na przykład poprzez konstrukcję miary
Jordana. 

\subsection{Dowód przez podobieństwo}

\begin{figure}[h]
\centering
\includegraphics[width=7cm]m
\caption{Dowód twierdzenia Pitagorasa przez podobieństwo}
\label{fig:obrazek m}
\end{figure}

Jest to jeden z dowodów podanych przez Euklidesa, wykorzystuje on podobieństwo
trójkątów. Zauważmy, że na rysunku obok trójkąty: duży - $\Delta$ABC, różowy - $\Delta$ADC i
niebieski - $\Delta$DBC są podobne. Niech |AB| = c, |BC| = a i |AC| = b. Można napisać
proporcje:
\begin{displaymath}
\frac{|DB|}{a} = \frac{a}{c}
\end{displaymath}
\begin{displaymath}
\frac{|AD|}{b} = \frac{b}{c}
\end{displaymath}

\indent Stąd:

\begin{center} 
a$^{2}$ = $c\cdot|DB|$\\
b$^{2}$ = $c\cdot|AD|$
\end{center}

\indent I po dodaniu stronami:

\begin{displaymath}
a^{2} + b^{2} = c\cdot |DB| + c\cdot |AD| = c (|DB|\cdot |AD|) = c^{2}
\end{displaymath}


\section{Twierdzenie odwrotne}

\subsection{Teza}
\begin{figure}[h]
\centering
\includegraphics[width=5cm]n
\caption{Odwrotne twierdzenie Pitagorasa}
\label{fig:obrazek n}
\end{figure}
\emph{Jeśli dane są trzy dodatnie liczby a,b i c takie, że $a^{2} + b^{2} = c^{2}$, to istnieje trójkąt o
bokach długości a,b i c, a kąt między bokami o długości a i b jest prosty.}
\\\\
\indent Najprawdopodobniej twierdzenie to wykorzystywane było w wielu starożytnych
kulturach Azji (Chinach, Indiach, Babilonii) i Egipcie do praktycznego wyznaczania kąta
prostego. Wystarczy bowiem zbudować trójkąt o bokach długości 3, 4 i 5 jednostek, aby
uzyskać kąt prosty między bokami o długościach 3 i 4. 

\subsection{Dowód}
Twierdzenie to można udowodnić na przykład metodą sprowadzenia do sprzeczności
lub przy pomocy twierdzenia cosinusów. 

\section{Uogólnienia}
Pewne uogólnienia twierdzenia Pitagorasa zostały podane już przez Euklidesa w jego
elementach: jeśli zbuduje się figury podobne na bokach trójkąta prostokątnego, to suma pól
powierzchni dwóch mniejszych będzie równa polu powierzchni największej figury. 

\subsection{Twierdzenie cosinusów}
Uogólnienie twierdzenia Pitagorasa na dowolne, niekoniecznie prostokątne, trójkąty
nosi nazwę twierdzenia cosinusów (\ref{eq:eps}) i znane było już w starożytności:
Jeśli w trójkącie o bokach długości \textit{a,b i c} oznaczyć przez $\gamma$ miarę kąta leżącego naprzeciw
boku \textit{c}, to prawdziwa jest równość: 

\begin{equation}
a^{2} + b^{2} - 2\textit{ab}cos\gamma = c^{2}
\label{eq:eps}
\end{equation}

\subsection{Twierdzenie Dijkstry o trójkątach}
Trywialny wniosek z twierdzenia cosinusów zgrabnie sformułował Edsger Dijkstra:\\\\
\emph{Jeżeli w dowolnym trójkącie naprzeciw boków długości a,b i c znajdują się odpowiednio kąty
$\alpha,\beta,\gamma,$ to zachodzi równość: }

\begin{displaymath}
sgn(\alpha+\beta-\gamma) = sgn(a^{2}+b^{2}-c^{2})
\end{displaymath}
\indent gdzie $sgn$ oznacza funkcje signum.

\newpage

\section{Bibliografia}
W poniższej tabeli (\ref{tabela}) znajdują się autorzy i prace wykorzystane w tym tekście.
\begin{table}[h]
\caption{Bibliografia}
\label{tabela}
\begin{center}
\begin{tabular}{|l|l|}
\hline
\textbf{Autor} & \textbf{Tytuł}\\ \hline
Szczepan Jeleński & Śladami Pitagorasa \\ \hline
Marek Piasecki & Wzór Pitagorasa \\ \hline
\end{tabular}
\end{center}
\end{table}



















 
\end{document}